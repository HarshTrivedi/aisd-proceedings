Automated/assisted scientific discovery is one of the most exciting emerging areas of AI, fueled by the promise - or at least hope - that language models can overcome some of the show-stopping obstacles of the past. At Ai2 we are pursuing this topic, developing both a sophisticated research assistant for humans, and prototyping several (semi-)autonomous scientific discovery systems. First I'll describe our research assistant with an extended example dialog, showing how the user can use tools to search the literature, identify relevant data, run software experiments, and analyze the results, illustrating the vision we are pursuing. Following this, I'll describe our (largely) autonomous prototypes, showing how they iteratively design and perform experiments on their own (for select computer science tasks: probing language models, building software agents, and improving transformer architectures). In particular, I'll highlight where they succeed what their fundamental limitations are. Finally I'll speculate on what it would take to go further, from today's systems that typically search a somewhat bounded and delineated space, to "big science" where research is conducted over many iterations and months, where the search spaces themselves are regularly redefined, and where predictive theories about the world gradually evolve.