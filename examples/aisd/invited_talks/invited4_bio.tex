Marinka Zitnik is an Associate Professor at Harvard in the Department of Biomedical Informatics. Dr. Zitnik is Associate Faculty at the Kempner Institue for the Study of Natural and Artificial Intelligence, Broad Institute of MIT and Harvard, and Harvard Data Science. Dr. Zitnik investigates foundations of AI to enhance scientific discovery and facilitate individualized diagnosis and treatment in medicine. Before joining Harvard, she was a postdoctoral scholar in Computer Science at Stanford University. She was also a member of the Chan Zuckerberg Biohub at Stanford. She received her bachelor’s degree, double majoring in computer science and mathematics, and then graduated with a Ph.D. in Computer Science from University of Ljubljana just three years later while also researching at Imperial College London, University of Toronto, Baylor College of Medicine, and Stanford University. This research received best paper and research awards from International Society for Computational Biology, Bayer Early Excellence in Science Award, Amazon Faculty Research Award, Google Faculty Research Scholar Award, Roche Alliance with Distinguished Scientists Award, Sanofi iDEA-iTECH Award, Rising Star Award in Electrical Engineering and Computer Science (EECS), and Next Generation Recognition in Biomedicine. Dr. Zitnik received the Kavli Fellowship by the US National Academy of Sciences and the Kaneb Fellowship award at Harvard Medical School. She also received the NSF CAREER Award. Dr. Zitnik is an ELLIS Scholar in the European Laboratory for Learning and Intelligent Systems (ELLIS) Society. She is a member of the Science Working Group at NASA Space Biology. She is also the 2025 Member of the Senior Common Room at Leverett House at Harvard University. Dr. Zitnik co-founded Therapeutics Data Commons and is the faculty lead of the AI4Science initiative. Dr. Zitnik is the recipient of the 2022 Young Mentor Award at Harvard Medical School.