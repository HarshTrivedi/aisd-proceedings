Welcome to AISD, the First Workshop on AI & Scientific Discovery, co-located with NAACL 2025 in Albuquerque, New Mexico. \\

Just as coding assistants have dramatically increased productivity for coding tasks over the last two years, researchers in the NLP community have begun to explore methods and opportunities ahead for creating scientific assistants that can help with the process of scientific discovery and increase the pace at which novel discoveries are made. Over the last year, language models have been used to create problem-general scientific discovery assistants that are not restricted to narrow problem domains or formulations. Such applications hold opportunities for assisting researchers in broad domains, or scientific reasoning more generally. Beyond assisting, a growing body of work has begun to focus on the prospect of creating largely autonomous scientific discovery agents that can make novel discoveries with minimal human intervention.
These recent developments highlight the possibility of rapidly accelerating the pace of scientific discovery in the near term. Given the influx of researchers into this expanding field, this workshop proposes to serve as a vehicle for bringing together a diverse set of perspectives from this quickly expanding subfield, helping to disseminate the latest results, standardize evaluation, foster collaboration between groups, and allow discussing aspirational goals for 2025 and beyond. This workshop welcomes and covers a wide range of topics, including (but not limited to): Literature-based Discovery, Agent-centered Approaches, Automated Experiment Execution, Automated Replication, Data-driven Discovery, Discovery in Virtual Environments, Discovery with Humans in the Loop, and Assistants for Scientific Writing. \\

A total of 7 papers appear in the proceedings. 24 papers were presented at the workshop itself, with the rest being submitted under two archival options: cross-submissions (Findings papers or those already presented at other venues, such as ICLR, EMNLP, NeurIPS, or the NAACL main conference), and regular non-archival submissions (unpublished work). The latter went through a normal peer review process. These papers can be found on the AISD website: https://ai-and-scientific-discovery.github.io/ \\

Six papers were featured as oral presentations. These papers represented a selection of strong work that the organizers felt would be of broad interest to workshop participants. In addition, we featured four invited talks: Heng Ji, Jure Leskovec, Peter Clark, and Marinka Zitniki.
We are thankful to all reviewers for their help in the selection of the program, for their readiness to engage in thoughtful discussions about individual papers, and for providing valuable feedback to the authors. We would also like to thank the NAACL workshop organizers for all the valuable help and support with the organizational aspects of the conference. Finally, we would like to thank all our authors and presenters for making this such an exciting event! \\

Peter Jansen, Bhavana Dalvi Mishra, Harsh Trivedi, Bodhisattwa Prasad Majumder, Tom Hope, Tushar Khot, Doug Downey, Eric Horvitz\\
AISD organizers